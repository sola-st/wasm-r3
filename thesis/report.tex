%-------------------------------------------------------------------------------------------------
%
%  Skeleton for bachelor and master thesis reports at the Software Lab
%-----------------------------------------------------------------------------------------
%
% USAGE:      compile with PDFLaTeX
%
% HISTORY:    - written by Sascha A. Stoeter <stoeter@iris.ethz.ch>, www.stoeter.com, 02.06.2004
%             - modified by Martin Probst, 18.08.2004
%             - extended and adapted for use at LST by Oliver Trachsel, 2007-08-21
%             - modified for use at Software Lab by Michael Pradel, Nov 2014
%             - modified for use at U Stuttgart by Michael Pradel, Sep 2020
%-------------------------------------------------------------------------------------------------
\documentclass[11pt,a4paper]{book}
\usepackage{solareport}

% -------------------------------------------------------------------------------------------------
% Add needed packages. Some generally useful packages are listed for
% your convenience.
% -------------------------------------------------------------------------------------------------
\usepackage{subfigure}                          % enable the use of subfigures
\usepackage[thickspace,thinqspace]{SIunits}     %
\usepackage[plainpages=false,pdfpagelabels]{hyperref}    % enable hyperlinks in pdf/ps Docs
\usepackage{listings}                           % to embed source code

% -------------------------------------------------------------------------------------------------
% Select type of thesis
% -------------------------------------------------------------------------------------------------
%\def\thesistype{Bachelor}
\def\thesistype{Master}

% -------------------------------------------------------------------------------------------------
% Set names
% -------------------------------------------------------------------------------------------------
\def\thesisauthor{John Doe}
\def\thesisadvisor{} % leave empty if directly advised by Michael Pradel
\def\studyprogram{Computer Science} % e.g., Computer Science or INFOTECH

% -------------------------------------------------------------------------------------------------
% Set dates
% -------------------------------------------------------------------------------------------------
\def\thesisstartdate{November 1, 2020} 
\def\thesisenddate{April 30, 2021} 


% -------------------------------------------------------------------------------------------------
% Beginning of the main document body
% -------------------------------------------------------------------------------------------------
\begin{document}

% -------------------------------------------------------------------------------------------------
% Front matter with title page, table of contents, and abstracts 
% -------------------------------------------------------------------------------------------------
\frontmatter

% Title page: set title
\thesistitlepage{Thesis Title}

% Abstract must not be longer than one page per language. English and
% German abstracts are mandatory.
\chapter*{Abstract}
Short summary of thesis.

\chapter*{Zusammenfassung}
Kurzfassung der Arbeit.

% Table of contents
\tableofcontents

% -------------------------------------------------------------------------------------------------
% Main document body
% -------------------------------------------------------------------------------------------------
\mainmatter
\chapter{Introduction}
\label{s:Introduction}
Explain scope and structure of report.

\chapter{Great Work}
\label{s:GreatWork}

This and the following chapters detail the original work.

\chapter{Examples}
\label{s:Examples}

This chapter provides some additional hints and examples for the
layout and style of the thesis. It is worthwhile to look at the source
file \verb|Examples.tex| for this appendix to understand how it was
created.

\section{Tables}

Tables are left justified and the caption appears on top as seen in
Table~\ref{t:Translations}.

\begin{table}[ht]
\centering
\begin{tabular}{ll}
\hline
\textbf{English} & \textbf{German}\\
\hline
cell phone       & Handy\\
Diet Coke        & Coca Cola light\\
\hline
\end{tabular}
\caption[Translations]{\label{t:Translations}Translations.}
\end{table}

\section{Figures}

Figure~\ref{f:SOLAlogo} shows a simple figure with a single picture
and Figure~\ref{f:SubfigureExample} shows a more complex figure
containing subfigures.

\begin{figure}[ht]
\centering
\includegraphics[width=.6\linewidth]{figures/SOLALogo}
\caption[SOLA logo]{\label{f:SOLAlogo}SOLA logo.}
\end{figure}

\begin{figure}[ht]
\centering
\subfigure[UStuttLogo]{\includegraphics[height=12mm]{figures/UStuttLogo}}\quad
\subfigure[SOLALogo]{\includegraphics[height=12mm]{figures/SOLALogo}}
\caption[Subfigure example]{\label{f:SubfigureExample}Two pictures as
  part of a single figure through the magic of the subfigure package.}
\end{figure}

\section{Units}

The SIUnits package provides nice spacing for units as demonstrated in
Table~\ref{t:SIUnits}. Use of the package also makes it easy to change
the style or even the unit text in the future.

\begin{table}[ht]
\centering
\begin{tabular}{ll}
\hline
\textbf{Output}   & \textbf{Command}\\
\hline
42m               & \verb|42m|\\
\unit{42}{\metre} & \verb|\unit{42}{\metre}|\\
42 m              & \verb|42 m|\\
\hline
\end{tabular}
\caption[Spacing for units]{\label{t:SIUnits}Spacing for units.}
\end{table}

\section{Source code}

The listings package provides tools to typeset source code
listings. It supports many programming languages and provides a lot of
formatting options.

\lstset{numbers=left, numberstyle=\tiny, stepnumber=1, numbersep=5pt}
\lstset{basicstyle=\ttfamily}
\lstset{frame=tb}

\begin{lstlisting}[float,caption=Example usage of the listing package,label=l:javaClass,language=Java]
class S {
   int f1 = 42;
   public S(int x) {
          f1 = x;
   }
}
\end{lstlisting}

Listing \ref{l:javaClass} shows an example listing. Code snippets can
also be inserted in normal text:
\verb$\lstinline|int f1 = 42;|$ gives \lstinline$int f1 = 42;$

\section{Miscellany}

\begin{description}

\item[Capitalization.] When referring to a named table (such as in the
  previous section), the word \emph{table} is capitalized. The same is
  true for figures, chapters and sections.

\item[Bibliography.] Use \verb|bibtex| to make your life easier and to
  produce consistently formatted entries.

\item[Contractions.] Avoid contractions. For instance, use ``do not''
  rather than ``don't.''

\item[Style guide.] A classic reference book on writing style is
  Strunk's \emph{The Elements of Style} \cite{Strunk-ElementsOfStyle}.

\end{description}

% -------------------------------------------------------------------------------------------------
% Appendices (if needed)
% -------------------------------------------------------------------------------------------------
\appendix
\chapter{Extra Stuff}
\label{s:ExtraStuff}

Additional material such as long mathematical derivations.

% -------------------------------------------------------------------------------------------------
% Bibliography
% -------------------------------------------------------------------------------------------------
\addcontentsline{toc}{chapter}{Bibliography}
\bibliography{report}

\chapter*{}

\vspace{-14em}
\textbf{Selbsst\"andigkeitserkl\"arung}\\

\noindent

Ich versichere, diese Arbeit selbstst\"andig verfasst zu haben. Ich habe keine anderen als die
angegebenen Quellen benutzt und alle w\"ortlich oder sinngem\"a{\ss} aus anderen Werken \"ubernommenen
Aussagen als solche gekennzeichnet. Weder diese Arbeit noch wesentliche Teile daraus waren bisher
Gegenstand eines anderen Pr\"ufungsverfahrens. Ich habe diese Arbeit bisher weder teilweise noch
vollst\"andig ver\"offentlicht.
Das elektronische Exemplar stimmt mit allen eingereichten Exemplaren \"uberein.

\vspace{8em}
\noindent\begin{tabular}{p{0.5\linewidth}p{0.4\linewidth}}
\rule{0.25\textwidth}{0.4pt} & \rule{0.4\textwidth}{0.4pt} \\
\textbf{Datum} & \textbf{Unterschrift}  \\
\end{tabular}


\end{document}
